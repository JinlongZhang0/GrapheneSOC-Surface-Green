\PaperTitle{从多等价Dirac锥体系到高Chern数量子反常霍尔效应:自旋轨道耦合和磁诱导下的石墨烯} % Article title
\Authors{JLZhang1996\textsuperscript{1*}} % Authors
\affiliation{

	\quad
	\textsuperscript{1} {GDPi}
	\qquad
	*:jlzhang1996@gmail.com
} % Author affiliation

\Abstract{
	\phantom{田田}石墨烯是非常典型的二维Dirac材料,由于良好的调控性质,在近些年被广泛关注。根据现有的能带反转(Band Inversion)的图像,本文研究了Rashba自旋轨道耦合效应和磁Zeeman场下,多Dirac锥材料实现高Chern数量子霍尔效应的可能性。由于石墨烯具有位于$\mathbf{K}$和$\mathbf{K'}$两个等价Dirac锥,该体系出现了$C=2$的量子反常霍尔效应。并且,利用求解迭代格林函数和开边界的格点哈密顿量,表明了局域在Zigzag边界处分别存在两条手性边缘态,最后讨论了这一特殊的边缘态在输运实验上的奇特性质。
}


\Keywords{\phantom{田田}多Dirac锥体系,石墨烯,高Chern量子反常霍尔效应}
% 如不需要关键词可直接删去花括号中内容

